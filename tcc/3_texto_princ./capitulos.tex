\setlength\abovedisplayskip{0pt} \setlength\belowdisplayskip{0pt}
\setlength\abovedisplayshortskip{0pt} \setlength\belowdisplayshortskip{0pt}

\chapter{Fundamenta\c{c}\~ao Te\'orica}
\label{cap1}

Neste cap�tulo ser�o discutidos e explicados todos os conceitos e 
fundamenta��o te�rica necess�rios para a compreens�o da proposta 
feita nesse trabalho, assim como a abordagem que ser� utilizada 
para a implementa��o de todos os componentes que juntos v�o compor 
o motor de jogos que ser� o objeto de estudo deste trabalho.

\section{Modelagem Orientada a Dados}

A modelagem orientada a dados (do ingles: \textit{Data Oriented Design}) � 
uma forma de codificar os programas que prop�e uma mudan�a no foco da 
implementa��o, ao inv�s de se focar no c�digo, o foco deve estar nos 
dados.

\section{Linguagem de Programa��o Rust}

\section{Biblioteca Matem�tica}

\subsection{Outros Conceitos Matem�ticos}

\section{Renderizador Gr�fico de Baixo N�vel}

\section{Malhas de Pol�gono}

\section{Texturas}

\section{Shaders}

\section{Ilumina��o}

\chapter{Trabalho Proposto}
\label{cap2}

\section{Componente Gr�fico}

\subsection{C�mera}

\subsection{Grafo de Cenas}

\subsection{Otimiza��es de Renderiza��o}

\section{Administrador de Recursos}

\chapter{Trabalhos Relacionados}
\label{cap3}

\chapter{Testes e Resultados Parciais}
\label{cap4}
