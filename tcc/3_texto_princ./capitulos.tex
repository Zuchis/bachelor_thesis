\setlength\abovedisplayskip{0pt} \setlength\belowdisplayskip{0pt}
\setlength\abovedisplayshortskip{0pt} \setlength\belowdisplayshortskip{0pt}

\chapter{Fundamenta\c{c}\~ao Te\'orica}
\label{cap1}

Neste capítulo serão discutidos e explicados todos os conceitos e 
fundamentação teórica necessários para a compreensão da proposta 
feita nesse trabalho, assim como a abordagem que será utilizada 
para a implementação de todos os componentes que juntos vão compor 
o motor de jogos que será o objeto de estudo deste trabalho.

\section{Modelagem Orientada a Dados}

A modelagem orientada a dados (do ingles: \textit{Data Oriented Design}) é 
uma forma de codificar os programas que propõe uma mudança no foco da 
implementação, ao invés de se focar no código, o foco deve estar nos 
dados.

\section{Linguagem de Programação Rust}

\section{Biblioteca Matemática}

\subsection{Outros Conceitos Matemáticos}

\section{Renderizador Gráfico de Baixo Nível}

\section{Malhas de Polígono}

\section{Texturas}

\section{Shaders}

\section{Iluminação}

\chapter{Trabalho Proposto}
\label{cap2}

\section{Componente Gráfico}

\subsection{Câmera}

\subsection{Grafo de Cenas}

\subsection{Otimizações de Renderização}

\section{Administrador de Recursos}

\chapter{Trabalhos Relacionados}
\label{cap3}

\chapter{Testes e Resultados Parciais}
\label{cap4}
