Devido � diferen�a de desempenho entre os processadores e o acesso � 
mem�ria principal, muitas aplica��es modernas t�m 
sofrido com um problema conhecido como gargalo do processador-mem�ria, no 
qual o processador fica ocioso aguardando dados serem buscados na mem�ria. 
Aplica��es gr�ficas, principalmente jogos eletr�nicos, s�o not�rias por 
sofrerem deste problema pela elevada quantidade de dados necess�rios para o
seu funcionamento. Este trabalho prop�e a implementa��o de um motor de jogos 
(\textit{game engine}) utilizando um conjunto de pr�ticas de programa��o, cuja 
premissa � escrever um c�digo que otimize a leitura da mem�ria atrav�s da 
correta estrutura��o de dados. Este conjunto de pr�ticas � conhecido como projeto 
orientado a dados, do ingl�s \textit{Data Oriented Design} (DOD), e possui potencial 
para amenizar esse problema do gargalo de desempenho. Para testar a efici�ncia do DOD, 
foram desenvolvidas duas aplica��es diferentes, uma utilizando DOD, e outra com 
programa��o orientada a objetos, considerada padr�o no mercado, e seus desempenho foram 
comparados. Os resultados mostram que o DOD � uma alternativa mais robusta e apresenta 
um desempenho mais favor�vel na maioria dos casos. Por fim foram feitas as conclus�es 
a respeito do DOD e sua utilidade.
