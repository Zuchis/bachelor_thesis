Devido � diferen�a de desempenho entre os processadores e o acesso � 
mem�ria principal, muitas aplica��es modernas t�m 
sofrido com um problema conhecido como gargalo do processador-mem�ria, no 
qual o processador fica ocioso aguardando dados serem buscados na mem�ria. 
Aplica��es gr�ficas, principalmente jogos eletr�nicos, s�o not�rias por 
sofrerem deste problema pela elevada quantidade de dados necess�rios para o
seu funcionamento. Este trabalho prop�e a implementa��o de um motor de jogos 
(\textit{engine}) utilizando um conjunto de pr�ticas de programa��o, a
 modelagem orientada a dados (MOD), com a finalidade de amenizar este problema. 
Para testar a efici�ncia da MOD, foram desenvolvidas duas aplica��es diferentes, uma 
utilizando a MOD, e outra com programa��o orientada a objetos, considerada padr�o no 
mercado, e seus desempenho foram comparados. Os resultados da compara��o foram 
discutidos e uma conclus�o foi feita para o trabalho.
