Devido � diferen�a de desempenho entre os microprocessadores e a velocidade de acesso 
� mem�ria principal, muitas aplica��es de computador modernas t�m sofrido com um problema 
conhecido como gargalo do processador-mem�ria, no qual o processador fica ocioso aguardando 
dados serem buscados na mem�ria. Aplica��es gr�ficas, principalmente jogos eletr�nicos, s�o 
not�rias por sofrerem deste problema pela demasiada quantidade de elementos diferentes que as 
constituem. Este trabalho prop�e a implementa��o de um motor de jogos (uma \textit{engine}) 
utilizando um conceito de estrutura��o de c�digo, a modelagem orientada a dados, com a 
finalidade de amenizar este problema de gargalo causado pelo acesso constante � mem�ria.
O trabalho ser� implementado na linguagem de programa��o Rust, linguagem a qual visa 
velocidade e uso eficiente e seguro da mem�ria.
