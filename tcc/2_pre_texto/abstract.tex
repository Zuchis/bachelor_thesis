Due to the performance difference between the processors and the access to the main 
memory, many modern applications have been suffering from a problem known as 
the processor-memory bottleneck, in which the processor stays idle while it waits for 
data to be fetched in the memory. Graphical applications, specially electronic games, are 
notorious for suffering this kind of problem because of the high amount of data 
for its functioning. This paper proposes the implementation of a game engine
using a set of programming practices, known as data oriented design (DOD), with the 
purpose of reducing this problem. In order to test the DOD efficiency, two applications 
were developed, one using the MOD, and the other one using object-oriented programming, 
which is considered the industry standard, and the performance of both were compared. 
The results were discussed and a conclusion was made.
