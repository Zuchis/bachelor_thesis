Due to the performance difference between the processors and the access to the main 
memory, many modern applications have been suffering from a problem known as 
the processor-memory bottleneck, in which the processor stays idle while it waits for 
data to be fetched in the memory. Graphical applications, specially electronic games, are 
notorious for suffering this kind of problem because of the high amount of data 
for its functioning. This work proposes the implementation of a game engine
using a set of programming practices, whose premise is to write code that optimizes the 
memory read through the correct data structuring. This set of practices is known as data 
oriented design (DOD), and it has the potential of reducing this performance bottleneck 
problem. In order to test the DOD efficiency, two applications were developed, one using 
DOD, and the other one using object-oriented programming, which is considered the 
industry standard, and the performance of both were compared. The results show that DOD is 
a more robust approach and shows a better performance in most of the cases. In the end 
conclusions were drawn about DOD and its utility.
