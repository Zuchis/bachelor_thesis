\documentclass{beamer}

\usepackage{graphicx,hyperref,udesc,url}
\usepackage[utf8]{inputenc}
\usepackage[T1]{fontenc}
\usepackage{booktabs}
%\usepackage[portugues]{babel}
\usepackage{amssymb}
\usepackage[utf8]{inputenc}
\usepackage[brazil]{babel}
\usepackage{csquotes}
\usepackage{listings}
\usepackage{amsmath}
\usepackage{amsthm}
\usepackage{mathtools}
\usepackage{verbatim}
%\usepackage[table,xcdraw]{xcolor}
\usepackage{multirow}

\title[]{Otimizações para Motores de Jogos Através de Modelagem Orientada a Dados}

\author[Vinicius Bruch Zuchi]{
    Vinicius Bruch Zuchi\\\medskip
    {\small \url{vinicius.b.zuchi@gmail.com}\\}}

\institute[UDESC]{
    Departamento de Ci\^encia da Computa\c{c}\~ao \\
    Centro de Ci\^encias e Tecnol\'ogicas\\
Universidade do Estado de Santa Catarina}

\begin{document}

\begin{frame}
    \titlepage
\end{frame}

\begin{frame}
    \frametitle{Sum\'ario}
    \tableofcontents
\end{frame}

\section{Recapitulando}

\frame{\tableofcontents[currentsection]}

\begin{frame}{Problema}
    \begin{itemize}
        \item Processadores possuem um alto poder computacional, porém o acesso à memória é relativamente lento
        \item Gargalo de desempenho na velocidade de acesso à memória
        \item Processador gasta muitos ciclos ociosos
    \end{itemize}
\end{frame}

\begin{frame}{Motor de Jogos}
    \begin{itemize}
        \item Separação entre a parte técnica e criativa do jogo
        \item Conjunto de componentes que interagem entre si
        \item A interação cria os loops de jogo (inputs, atualização e renderização)
        \item Motivação: modularização, rápida prototipagem e desenvolvimento, e portabilidade
    \end{itemize}
\end{frame}

\begin{frame}{Motor de Jogos}
    \begin{itemize}
        \item Jogos são notórios por possuirem um grande volume de dados
        \item Muitas entidades que necessitam de atualização e processamento
        \item Muitos dados sendo lidos da memória
        \item Se o agrupamento de dados não está coerente: poluição de cache
    \end{itemize}
\end{frame}

\begin{frame}{Como Solucionar}
    \begin{itemize}
        \item Não há como aumentar a capacidade dos caches...
        \item O que resta aos desenvolvedores é organizar o código e estruturas de dados
              de tal forma a otimizar a leitura dos dados para uma linha de cache.
    \end{itemize}
\end{frame}

\section{Modelagem Orientada a Dados}

\frame{\tableofcontents[currentsection]}

\begin{frame}{Origem e Popularização}
    \begin{itemize}
        \item Introduzido por John A. Sharp em 1980
            \begin{itemize}
                \item Seu objetivo era aprimorar o tempo de processamento e eficiência da memória
                \item Organização dos dados para permitir uma leitura sequencial destes
                \item Separação de dados quentes e frios
            \end{itemize}
        \item Popularizado como \textit{Data Oriented Design} por Mike Acton em 2009
        \item Surgimento de motores de jogos utilizando os conceitos de DOD
    \end{itemize}
\end{frame}

\begin{frame}{Principais Ideias}
    \begin{itemize}
        \item Primeiro passo de um projeto: pense primeiro nos dados que serão utilizados, e depois no código
        \item Hierarquia de classes não é importante, mas os padrões de acesso aos dados é.
        \item Pense no fluxo de dados da aplicação, isto é: analisar como os dados são acessados, 
              transformados, e sua finalidade.
        \item Onde há um, há muitos (O fluxo de dados ocorre para várias entidades, e não somente uma).
        \item Esteja ciente do \textit{overhead} de funções virtuais, ponteiros para funções, e ponteiros para 
            métodos membros de classe.
    \end{itemize}
\end{frame}

\begin{frame}[t]{Como Armazenar os Dados}
    \begin{figure}
        \begin{minipage}[b]{0.35\textwidth}
        Arrays of Structures (AoS)
        \centering
            \includegraphics[width=\textwidth]{figuras/aosscheme}
        \end{minipage}
        \hspace{1.5cm}
        \begin{minipage}[b]{0.35\textwidth}
        \centering
        Structures of Arrays (SoA)
            \includegraphics[width=\textwidth]{figuras/soascheme}
        \end{minipage}
    \end{figure}
\end{frame}

\begin{frame}{Como Armazenar os Dados}
    \begin{minipage}[b]{0.4\textwidth}
        Arrays of Structures (AoS)
        \begin{itemize}
            \item Esquema utilizado em POO
            \item Cada instância de uma classe/entidade possui todas as propriedades desta
            \item Pode levar a estruturas com dados poucos relacionados
        \end{itemize}
    \end{minipage}
    \hspace{1.5cm}
    \begin{minipage}[b]{0.4\textwidth}
        Structures of Arrays (SoA)
        \begin{itemize}
            \item Uma estrutura representa todos os objetos
            \item Propriedades separadas em \textit{arrays} diferentes
            \item Se torna mais ineficiente quando há muito endereçamento com índices
        \end{itemize}
    \end{minipage}
\end{frame}

\begin{frame}{Como Armazenar os Dados}
    \begin{itemize}
        \item Não há um melhor esquema, depende de cada caso em específico
        \item Podem ser utilizados em conjunto (como foi feito neste trabalho)
    \end{itemize}
\end{frame}

\begin{frame}{Branching e Prefetching}
    \begin{itemize}
        \item Duas motivações para o uso da MOD é evitar \textit{branching} 
              e facilitar \textit{prefetching} de dados e instruções
        \item \textit{Branching} ocorre com a mudança no fluxo de execução (condicionais)
        \item O \textit{prefetching} é facilitado pelo processamento sequencial, previsível 
            dos dados, e sem interrupções no fluxo.
    \end{itemize}
\end{frame}

\section{O Trabalho Desenvolvido}

\frame{\tableofcontents[currentsection]}

\begin{frame}{Motivação}
    \centering
    \LARGE{Verificar a eficiência da modelagem orientada a dados comparando-a 
    com a abordagem tradicional orientada a objetos}
\end{frame}

\begin{frame}{Aplicações Desenvolvidas}
    \begin{itemize}
        \item Duas aplicações foram desenvolvidas, uma com programação orientada a objetos (versão OO)
            e outra com modelagem orientada a dados (versão OD)
        \item Nem todos os componentes são diferentes
        \item Aplicações foram testadas em dois cenários diferentes
        \item Resultados gerados incrementando-se a quantidade de objetos em cena
    \end{itemize}
\end{frame}

\begin{frame}{Etapas de Conversão}
    \begin{itemize}
        \item Análise dos dados da aplicação e suas finalidades
        \item Análise do fluxo destes dados
        \item Conversão das estruturas de dados
        \item Conversão dos métodos
    \end{itemize}
\end{frame}

\begin{frame}{Diferenças Entre as Abordagens}
    \begin{itemize}
        \item As mudanças estão nas classes cujos métodos são executados por todas as instâncias por frame
        \item Possuem o processamento mais intenso, diferença é mais aparente
    \end{itemize}
\end{frame}

\begin{frame}{Diferenças Entre as Abordagens: Estruturas}
    \begin{figure}
    \centering
        \begin{minipage}[b]{0.35\textwidth}
            \includegraphics[width=\textwidth]{figuras/shaderood}
        \end{minipage}
        \hspace{1.5cm}
        \begin{minipage}[b]{0.35\textwidth}
            \includegraphics[width=\textwidth]{figuras/shaderdod}
        \end{minipage}
    \end{figure}
\end{frame}

\begin{frame}{Diferenças Entre as Abordagens: Estruturas}
    \begin{figure}
    \centering
        \begin{minipage}[b]{0.35\textwidth}
            \includegraphics[width=\textwidth]{figuras/meshood}
        \end{minipage}
        \hspace{1.5cm}
        \begin{minipage}[b]{0.35\textwidth}
            \includegraphics[width=\textwidth]{figuras/meshdod}
        \end{minipage}
    \end{figure}
\end{frame}

\begin{frame}{Diferenças Entre as Abordagens: Estruturas}
    \begin{figure}
    \centering
        \begin{minipage}[b]{0.35\textwidth}
            \includegraphics[width=\textwidth]{figuras/objectood}
        \end{minipage}
        \hspace{1.5cm}
        \begin{minipage}[b]{0.35\textwidth}
            \includegraphics[width=\textwidth]{figuras/objectdod}
        \end{minipage}
    \end{figure}
\end{frame}

\begin{frame}{Diferenças Entre as Abordagens: Estruturas}
    \begin{figure}
    \centering
        \begin{minipage}[b]{0.35\textwidth}
            \includegraphics[width=\textwidth]{figuras/nodeood}
        \end{minipage}
        \hspace{1.5cm}
        \begin{minipage}[b]{0.35\textwidth}
            \includegraphics[width=\textwidth]{figuras/nodedod}
        \end{minipage}
    \end{figure}
\end{frame}

\begin{frame}[t]{Método de Renderização versão OO}
    \begin{figure}[h!]
        \centering
        \includegraphics[width =.8\textwidth]{figuras/ooddraw}
        \par\medskip
    \end{figure}
\end{frame}

\begin{frame}{Método de Renderização versão OD}
    \begin{figure}[h!]
        \centering
        \includegraphics[width =\textwidth]{figuras/doddraw}
        \par\medskip
    \end{figure}
\end{frame}

\begin{frame}{Método de Atualização versão OO}
    \begin{figure}[h!]
        \centering
        \includegraphics[width =\textwidth]{figuras/oodupdate}
        \par\medskip
    \end{figure}
\end{frame}

\begin{frame}{Método de Atualização versão OD}
    \begin{figure}[h!]
        \centering
        \includegraphics[width =\textwidth]{figuras/dodupdate}
        \par\medskip
    \end{figure}
\end{frame}

\begin{frame}{Detecção de Colisões versão OO}
    \begin{figure}[h!]
        \centering
        \includegraphics[width =\textwidth]{figuras/oodcolision}
        \par\medskip
    \end{figure}
\end{frame}

\begin{frame}{Detecção de Colisões versão OD}
    \begin{figure}[h!]
        \centering
        \includegraphics[width =\textwidth]{figuras/dodcolision}
        \par\medskip
    \end{figure}
\end{frame}

\section{Análises e Resultados}

\frame{\tableofcontents[currentsection]}

\begin{frame}{Cenários Desenvolvidos}
    \begin{itemize}
        \item Problema A: Renderização de objetos sem hierarquia (1 nível)
        \item Problema B: Renderização de objetos com 4 níveis de hierarquia
    \end{itemize}
\end{frame}

\begin{frame}{Cenários Desenvolvidos}
    \begin{figure}[h]
        \centering
        \includegraphics[width =.45\textwidth]{figuras/problemBscheme}
        \par\medskip
    \end{figure}
\end{frame}

\begin{frame}{Problema A: Draw}
    \begin{figure}[h!]
        \centering
        \includegraphics[width =.8\textwidth]{figuras/drawv1graph}
        \par\medskip
    \end{figure}
\end{frame}

\begin{frame}{Problema A: Update}
    \begin{figure}[h!]
        \centering
        \includegraphics[width =.8\textwidth]{figuras/updatev1graph}
        \par\medskip
    \end{figure}
\end{frame}

\begin{frame}{Problema A: Collision}
    \begin{figure}[h!]
        \centering
        \includegraphics[width =.8\textwidth]{figuras/colisionv1graph}
        \par\medskip
    \end{figure}
\end{frame}

\begin{frame}{Problema A: FPS}
    \begin{figure}[h!]
        \centering
        \includegraphics[width =.8\textwidth]{figuras/fpsv1}
        \par\medskip
    \end{figure}
\end{frame}

\begin{frame}{Problema B: Draw}
    \begin{figure}[h!]
        \centering
        \includegraphics[width =.8\textwidth]{figuras/drawv2graph}
        \par\medskip
    \end{figure}
\end{frame}

\begin{frame}{Problema B: FPS}
    \begin{figure}[h!]
        \centering
        \includegraphics[width =.8\textwidth]{figuras/fpsv2}
        \par\medskip
    \end{figure}
\end{frame}

\section{Conclusão}

\frame{\tableofcontents[currentsection]}

\begin{frame}{Desafios}
    \begin{itemize}
        \item Pouco material dispnível nas fontes pesquisadas
        \item Determinar a melhor maneira de armazenar os dados
        \item Preocupação constante com o padrão de acesso à memória
        \item Não pensar em classes e o relacionamento entre elas 
            não é intuitivo
    \end{itemize}
\end{frame}

\begin{frame}{Conclusões a respeito de MOD}
    \begin{itemize}
        \item MOD não lhe garante aumento no desempenho, porém 
            proporciona um bom controle sobre este
        \item Quanto mais propriedades de uma entidade, mais 
            ineficiente se torna (problema do endereçamento por 
            índice)
        \item Mudança para métodos de processamento sequencial mostrou 
            ser uma alternativa muito melhor
        \item Não há um melhor esquema de armazenamento (AoS e SoA), 
            o ideal é combiná-los.
    \end{itemize}
\end{frame}

\begin{frame}{Fim}
    \centering
    \LARGE{Fim!}
\end{frame}

\end{document}
